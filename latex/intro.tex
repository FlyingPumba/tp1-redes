En este trabajo práctico nos proponemos analizar redes de información con el objetivo de puntualizar diversos aspectos analíticos de las mismas. Para lograr tal fin, utilizamos la herramienta \textit{Scapy}, la cual nos provee de una función llamada \texttt{sniff}. Esta función nos permite capturar todos los paquetes de la red en la cual nos encontramos durante un período de tiempo determinado.

Basándonos en la teoría de la información de Shannon y con los paquetes obtenidos, definimos tres fuente de información, $S$, $S_{1}$ y $S_{2}$, donde:

\begin{itemize}
  \item $S = \{s_{1} \dots s_{n}\}$, donde $s_{i}$ es el valor del campo \emph{type} de cada paquete \texttt{Ethernet} de capa 2.
  \item $S_{1} = \{s_{1} \dots s_{n}\} $, donde $s_i$ es el valor del campo \emph{destino} (IP) de cada paquete de
  tipo \texttt{ARP} Who-Has.
  \item $S_{2} = \{s_{1} \dots s_{n}\} $, donde $s_i$ es el valor del campo \emph{fuente} (IP) de cada paquete de
  tipo \texttt{ARP} Who-Has.
\end{itemize}

Es necesario aclarar que los paquetes \texttt{ARP}, el cual es un protocolo de capa 2.5, se utilizan para encontrar direcciones \texttt{MAC} (direcciones de capa 2) asociadas a cada \texttt{IP} (direcciones de capa 3). El procedimiento por el cual un host envía un paquete \texttt{ARP} se inicia cuando el host quiere enviar un paquete a una dirección IP, la cual se encuentra dentro de su red local, y no esta listada dentro de su tabla de traducciones de direcciones \texttt{MAC-IP}. Luego el host envía un paquete \texttt{ARP} de tipo Who-Has broadcast, dentro de su red local, para determinar la dirección \texttt{MAC} del host destino. El host con la dirección \texttt{IP} solicitada responde únicamente al host fuente con su dirección \texttt{MAC}, utilizando un paquete \texttt{ARP} de tipo Is-At.

A partir de la definición de las fuentes, lo que nos importa es determinar la cantidad de información $I(s_{i}) = -log_2(P(s_{i}))$ que aporta cada símbolo de la fuente, donde $Ps_{i}$ es la probabilidad de ocurrencia del símbolo $s_{i}$, la cual se obtiene mediante el cociente entre la cantidad de apariciones de $s_{i}$ en los paquetes y la cantidad total de paquetes capturados.

Luego vamos a comparar cada $I(s_{i})$ con la entropía de la fuente, $H(S_{i}) = \sum\limits_{s \in S_{i}} P(s) * I(s)$, para observar la presencia de \emph{protocolos distinguidos}, en el caso de la fuente $S$, o \emph{nodos distinguidos}, en el caso de la fuente $S_{1}$ y $S_{2}$. La noción de elemento distinguido la definimos como aquel símbolo cuya información es menor a la entropía de la fuente, es decir que la cantidad de información que aporta ese símbolo es baja, ya que el símbolo es predecible. La finalidad de este análisis parte de la idea de que una fuente cualquiera, $S$, sin perdida de información, satisface la ecuación $H(S) \leq L(C)$, donde $L(C)$ es el largo de la codificación $C$ de los símbolos de $S$. Para lograr que la codificación $C$ sea óptima, debemos codificar los símbolos distinguidos con menos bits que el resto de los símbolos de la fuente. Este tipo de análisis nos permite determinar que tan comprensible es una fuente, ya que cuanto menor sea su entropía mayor es su capacidad de comprensión.

Otro aspecto que vamos a analizar de las redes de información es la incidencia de los paquetes \texttt{ARP} en las mismas. Dado que estos paquetes son de control, es decir no transportan datos, impactan negativamente en el $throughput$ de una red, siendo $throughput$ el volumen de información neta que fluye a través de una red. Este análisis nos permitirá definir que red es mas eficiente en términos de datos transportados.
