En este trabajo práctico nos proponemos analizar redes de información con el objetivo de puntualizar diversos aspectos analiticos de las mismas. Para lograr tal fin, utilizamos la herramienta \textit{Scapy}, la cual nos provee de una función llamada \texttt{sniff}. Esta función nos permite capturar todos los paquetes de la red en la cual nos encontramos durante un período de tiempo determinado.

Basandonos en la teoría de la información de Shannon y con los paquetes obtenidos, definimos tres fuente de información, $S$, $S_{1}$ y $S_{2}$, donde:

\begin{itemize}
  \item $S = \{s_{1} \dots s_{n}\}$, donde $s_{i}$ es el valor del campo \emph{type} de cada paquete.
  \item $S_{1} = \{s_{1} \dots s_{n}\} $, donde $s_i$ es el valor del campo \emph{destino} (IP) cada paquete de
  de tipo ARP Who-Has.
  \item $S_{2} = \{s_{1} \dots s_{n}\} $, donde $s_i$ es el valor del campo \emph{fuente} (IP) cada paquete de
  de tipo ARP Who-Has.
\end{itemize}

Es necesario aclarar que los paquetes \texttt{ARP}, el cual es un protocolo de capa 2.5, se utilizan para encontrar direcciones \texttt{MAC} (direcciones de capa 2) asociadas a cada \texttt{IP} (direcciones de capa 3).

El procedimiento por el cual un host envía un paquete \texttt{ARP} se inicia cuando el host quiere enviar un paquete a una dirección IP, la cual se encuentra dentro de su red, y no esta listada dentro de su tabla de traducciones de direcciones \texttt{MAC-IP}. Luego el host envía un paquete who-has broadcast para determinar la dirección \texttt{MAC} del host destino. El host con la dirección \texttt{IP} solicitada responde unicamente al host fuente con su direccion \texttt{MAC}, utilizando un paquete \texttt{ARP} de tipo is-at.
