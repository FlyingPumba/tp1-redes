La problematica de este trabajo práctico fue implementar una herramienta para escuchar a la red y analisar los datos que obteniamos. A la vez, con esos datos propusimos diferentes
fuentes de información para las cuales calculamos la incidencia de cada simbolo de la fuente(mayoritariamente distinguidos por protocolos y nodos) y la entropía entre otras 
cosas. 

Usamos el lenguaje de programación Python y su librería Scapy para llevar a cabo el sniffeo de la red. 

Aprovechamos estas herramientas para realizar varios experimentos en distintos lugares y condiciones de las redes. A partir de los datos adquiridos generamos gráficos que representan
el comportamiento de la red y discutimos sus caracteristicas.