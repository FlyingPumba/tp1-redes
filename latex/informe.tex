\documentclass[a4paper]{article}
\usepackage[spanish]{babel}
\usepackage[utf8]{inputenc}
\usepackage{fancyhdr}
\usepackage{charter} % tipografia
%\usepackage{graphicx}
\usepackage[pdftex]{graphicx}
\usepackage{bm} % bold font in math mode
\usepackage{sidecap}
\usepackage{caption}
\usepackage{subcaption}
\usepackage{booktabs}
\usepackage{makeidx}
\usepackage{float}
\usepackage{amsmath, amsthm, amssymb}
\newtheorem{theorem}{Teorema}
\newtheorem{customthm}{Teorema}
\newtheorem{corollary}{Corolario}[theorem]
\newtheorem{proposition}[theorem]{Proposición}
\newtheorem{innercustomlemma}{Lemma}
\newenvironment{customlemma}[1]
  {\renewcommand\theinnercustomlemma{#1}\innercustomlemma}
  {\endinnercustomlemma}
\usepackage{amsfonts}
\usepackage{sectsty}
\usepackage{wrapfig}
\usepackage{listings}
\usepackage{hyperref} % links
\usepackage{algorithm} %http://www.ctan.org/pkg/algorithms
\usepackage{algorithmic}
\usepackage[usenames,dvipsnames]{xcolor}
\usepackage{pgfplots}
\usepackage{tabularx} % tablas copadas
% \usepackage{pgfplotstable}
% custom
\usepackage{color} % para snipets de codigo coloreados
\usepackage{fancybox} % para el sbox de los snipets de codigo
\definecolor{litegrey}{gray}{0.94}
% \newenvironment{sidebar}{%
% \begin{Sbox}\begin{minipage}{.85\textwidth}}%
% {\end{minipage}\end{Sbox}%
% \begin{center}\setlength{\fboxsep}{6pt}%
% \shadowbox{\TheSbox}\end{center}}
% \newenvironment{warning}{%
% \begin{Sbox}\begin{minipage}{.85\textwidth}\sffamily\lite\small\RaggedRight}%
% {\end{minipage}\end{Sbox}%
% \begin{center}\setlength{\fboxsep}{6pt}%
% \colorbox{litegrey}{\TheSbox}\end{center}}

%\newenvironment{codesnippet}{%
%\begin{Sbox}\begin{minipage}{\linewidth-2\fboxsep-2\fboxrule-4pt}\sffamily\small}%
%{\end{minipage}\end{Sbox}%
%\begin{center}%
%\colorbox{litegrey}{\TheSbox}\end{center}}

% \newenvironment{codesnippet}{\VerbatimEnvironment%
%   \noindent
%   %{\columnwidth-\leftmargin-\rightmargin-2\fboxsep-2\fboxrule-4pt}
%   \begin{Sbox}
%   \begin{minipage}{\linewidth-2\fboxsep-2\fboxrule-4pt}
%   \begin{Verbatim}
% }{%
%   \end{Verbatim}
%   \end{minipage}
%   \end{Sbox}%
%   \colorbox{litegrey}{\TheSbox}
% }

\newenvironment{codesnippet}{%
  \noindent
  %      {\columnwidth-\leftmargin-\rightmargin-2\fboxsep-2\fboxrule-4pt}
  \begin{Sbox}
  \begin{minipage}{\linewidth}
  \begin{lstlisting}
}{
  \end{lstlisting}
  \end{minipage}
  \end{Sbox}%
  \colorbox{litegrey}{\TheSbox}
}

\usepackage{fancyhdr}
\pagestyle{fancy}
%\renewcommand{\chaptermark}[1]{\markboth{#1}{}}
\renewcommand{\sectionmark}[1]{\markright{\thesection\ - #1}}
\fancyhf{}
\fancyhead[LO]{Sección \rightmark} % \thesection\
\fancyfoot[LO]{\small{Iv\'an Arcuschin, Mart\'in Jedwabny, Jos\'e Massigoge, Iv\'an Pondal}}
\fancyfoot[RO]{\thepage}
\renewcommand{\headrulewidth}{0.5pt}
\renewcommand{\footrulewidth}{0.5pt}
\setlength{\hoffset}{-0.8in}
\setlength{\textwidth}{16cm}
%\setlength{\hoffset}{-1.1cm}
%\setlength{\textwidth}{16cm}
\setlength{\headsep}{0.5cm}
\setlength{\textheight}{25cm}
\setlength{\voffset}{-0.7in}
\setlength{\headwidth}{\textwidth}
\setlength{\headheight}{13.1pt}
\renewcommand{\baselinestretch}{1.1} % line spacing

% -------------------- COMANDOS ESPECIALES ------------------------------

\newcommand{\calcular}[2]{\pgfmathtruncatemacro{#1}{#2}}

\pgfplotsset{
  filter params/.style n args={4}{
      x filter/.code={
          \edef\tempa{\thisrow{#1}}
          \edef\tempb{#2}
          \edef\tempc{\thisrow{#3}}
          \edef\tempd{#4}
          \ifx\tempa\tempb
            \ifx\tempc\tempd
            \else
              \def\pgfmathresult{inf}
            \fi
          \else
            \def\pgfmathresult{inf}
          \fi
      }
  }
}

\newcommand{\graficarDatos}[6]{
  \begin{tikzpicture}
  \begin{axis}[
      title={#1},
      xlabel={#2},
      ylabel={#3},
      scaled x ticks=false,
      scaled y ticks=false,
      scale=0.5
  ]
  \addplot[only marks, color=black] table[x=#4,y=#5]{#6};
  \end{axis}
  \end{tikzpicture}
}

\newcommand{\graficarDatosPlus}[7]{
  \begin{tikzpicture}
  \begin{axis}[
      title={#1},
      xlabel={#2},
      ylabel={#3},
      scaled x ticks=false,
      scaled y ticks=false,
      width=0.6\textwidth,
      #7
  ]
  \addplot[only marks, color=black] table[x=#4,y=#5]{#6};
  \end{axis}
  \end{tikzpicture}
}

\makeatletter
\pgfplotsset{
    groupplot xlabel/.initial={},
    every groupplot x label/.style={
        at={($({group c1r\pgfplots@group@rows.west}|-{group c1r\pgfplots@group@rows.outer south})!0.5!({group c\pgfplots@group@columns r\pgfplots@group@rows.east}|-{group c\pgfplots@group@columns r\pgfplots@group@rows.outer south})$)},
        anchor=north,
    },
    groupplot ylabel/.initial={},
    every groupplot y label/.style={
            rotate=90,
        at={($({group c1r1.north}-|{group c1r1.outer
west})!0.5!({group c1r\pgfplots@group@rows.south}-|{group c1r\pgfplots@group@rows.outer west})$)},
        anchor=south
    },
    execute at end groupplot/.code={%
      \node [/pgfplots/every groupplot x label]
{\pgfkeysvalueof{/pgfplots/groupplot xlabel}};
      \node [/pgfplots/every groupplot y label]
{\pgfkeysvalueof{/pgfplots/groupplot ylabel}};
    },
    group/only outer labels/.style =
{
group/every plot/.code = {%
    \ifnum\pgfplots@group@current@row=\pgfplots@group@rows\else%
        \pgfkeys{xticklabels = {}, xlabel = {}}\fi%
    \ifnum\pgfplots@group@current@column=1\else%
        \pgfkeys{yticklabels = {}, ylabel = {}}\fi%
}
}
}

\def\endpgfplots@environment@groupplot{%
    \endpgfplots@environment@opt%
    \pgfkeys{/pgfplots/execute at end groupplot}%
    \endgroup%
}
\makeatother

\newcommand{\barGraphExp}[2]{
    \begin{tikzpicture}
    \begin{axis}[
        xlabel={Implementación},
    	ylabel={Tiempo de ejecución (clocks)},
        legend style={at={(1.4,1.0)}},
        ybar,
        scaled ticks=false,
        width=0.5\textwidth,
        height=0.5\textwidth,
        tickpos=left,
        xtick=\empty,
        ytick align=inside,
        xtick align=inside,
    	enlargelimits=0.05,
        bar width=16,
    ]
    % How to process each item:
    \renewcommand*{\do}[1]{\addplot+[color=black] table[x=n, y=##1]{datos/datos_blur.dat};}
    % Process list:
    \docsvlist{#2}
    \legend{#2}
    \end{axis}
    \end{tikzpicture}
}

\newcommand{\graficarDatosExp}[6]{
  \begin{tikzpicture}
  \begin{axis}[
      title={#1},
      xlabel={#2},
      ylabel={#3},
      scaled x ticks=false,
      scaled y ticks=false,
      enlargelimits=0.05,
      width=0.5\textwidth,
      height=0.5\textwidth
  ]
  \addplot[color=black] table[x=#5,y=#6]{#4};
  % \renewcommand*{\do}[1]{\addplot table[x=#5,y=##1]{#4};}
  % %     % Process list:
  % \docsvlist{#6}
  % \legend{#6}
  \end{axis}
  \end{tikzpicture}
}

% ------------------------------------------------------------------------

% \setcounter{secnumdepth}{2}
\usepackage{underscore}
\usepackage{kbordermatrix}% Matrix column labels
\usetikzlibrary{arrows,shapes}
\usepackage{tkz-graph}
\usepackage{caratula}
\usepackage{url}
\lstset{
    language=C++,
    basicstyle=\ttfamily,
    keywordstyle=\color{blue}\ttfamily,
    stringstyle=\color{red}\ttfamily,
    commentstyle=\color{ForestGreen}\ttfamily,
    morecomment=[l][\color{magenta}]{\#},
    literate={á}{{\'a}}1 {ó}{{\'o}}1 {é}{{\'e}}1 {í}{{\'i}}1 {ú}{{\'u}}1 {Á}{{\'A}}1 {Í}{{\'I}}1 {É}{{\'E}}1 {Ú}{{\'U}}1 {Ó}{{\'O}}1 {\ \ }{{\ }}1,
	breaklines=true,
	tabsize=2
}

\DeclareUnicodeCharacter{2212}{-}

% *********************** %
\usepackage{tikz}
\usetikzlibrary{graphs}
\usetikzlibrary{calc}
\usetikzlibrary{arrows}
\usetikzlibrary{matrix}
% Otros
\usepackage{arrayjobx}
\usepackage{enumitem}
\usepackage{multicol}
\usepackage{natbib}
\usepackage{etoolbox}
\usepackage{listingsutf8}
\lstset{inputencoding=utf8/latin1}
\usepackage{fancyvrb}
\usepackage{pgfplotstable}
\usepackage{float}
\newcommand{\subscript}[2]{$#1 _ #2$}


% ******************************************************** %
\begin{document}
\thispagestyle{empty}
\materia{Teoría de las Comunicaciones}
\submateria{Primer Cuatrimestre de 2016}
\titulo{Trabajo Práctico I}
%\subtitulo{Grupo: }
\integrante{Iv\'an Arcuschin}{678/13}{iarcuschin@gmail.com}
\integrante{Federico De Rocco}{408/13}{fede.183@hotmail.com}
\integrante{Mart\'in Jedwabny}{885/13}{martiniedva@gmail.com}
\integrante{Jos\'e Massigoge}{954/12}{jmmassigoge@gmail.com}
\maketitle
% no footer on the first page
\thispagestyle{empty}
\newpage

\tableofcontents

\newpage
\section{Introducción}
En este trabajo práctico nos proponemos analizar redes de información con el objetivo de puntualizar diversos aspectos analíticos de las mismas. Para lograr tal fin, desarrolamos una herramienta a partir de la librería \textit{Scapy}, la cual nos provee de una función llamada \texttt{sniff} y el software Wireshark. Ambos nos permitieron capturar todos los paquetes visibles de la red en la cual nos encontramos durante un período de tiempo determinado.

Basándonos en la teoría de la información de Shannon y con los paquetes obtenidos, definimos tres fuente de información, $S$, $S_{1}$ y $S_{2}$, donde:

\begin{itemize}
  \item $S = \{s_{1} \dots s_{n}\}$, donde $s_{i}$ es el valor del campo \emph{type} de cada paquete \texttt{Ethernet} de capa 2.
  \item $S_{1} = \{s_{1} \dots s_{n}\} $, donde $s_i$ es el valor del campo \emph{destino} (IP) de cada paquete de
  tipo \texttt{ARP} Who-Has.
\end{itemize}

Es necesario aclarar que los paquetes \texttt{ARP}, el cual es un protocolo de capa 2.5, se utilizan para encontrar direcciones \texttt{MAC} (direcciones de capa 2) asociadas a cada \texttt{IP} (direcciones de capa 3). El procedimiento por el cual un host envía un paquete \texttt{ARP} se inicia cuando el host quiere enviar un paquete a una dirección IP, la cual se encuentra dentro de su red local, y no esta listada dentro de su tabla de traducciones de direcciones \texttt{MAC-IP}. Luego el host envía un paquete \texttt{ARP} de tipo Who-Has broadcast, dentro de su red local, para determinar la dirección \texttt{MAC} del host destino. El host con la dirección \texttt{IP} solicitada responde únicamente al host fuente con su dirección \texttt{MAC}, utilizando un paquete \texttt{ARP} de tipo Is-At.

A partir de la definición de las fuentes, lo que nos importa es determinar la cantidad de información $I(s_{i}) = -log_2(P(s_{i}))$ que aporta cada símbolo de la fuente, donde $Ps_{i}$ es la probabilidad de ocurrencia del símbolo $s_{i}$, la cual se obtiene mediante el cociente entre la cantidad de apariciones de $s_{i}$ en los paquetes y la cantidad total de paquetes capturados.

Luego vamos a comparar cada $I(s_{i})$ con la entropía de la fuente, $H(S_{i}) = \sum\limits_{s \in S_{i}} P(s) * I(s)$, para observar la presencia de \emph{protocolos distinguidos}, en el caso de la fuente $S$, o \emph{nodos distinguidos}, en el caso de la fuente $S_{1}$ y $S_{2}$. La noción de elemento distinguido la definimos como aquel símbolo cuya información es menor a la entropía de la fuente, es decir que la cantidad de información que aporta ese símbolo es baja, ya que el símbolo es predecible. La finalidad de este análisis parte de la idea de que una fuente cualquiera, $S$, sin perdida de información, satisface la ecuación $H(S) \leq L(C)$, donde $L(C)$ es el largo de la codificación $C$ de los símbolos de $S$. Para lograr que la codificación $C$ sea óptima, debemos codificar los símbolos distinguidos con menos bits que el resto de los símbolos de la fuente. Este tipo de análisis nos permite determinar que tan comprensible es una fuente, ya que cuanto menor sea su entropía mayor es su capacidad de comprensión.

Otro aspecto que vamos a analizar de las redes de información es la incidencia de los paquetes \texttt{ARP} en las mismas. Dado que estos paquetes son de control, es decir no transportan datos, impactan negativamente en el $throughput$ de una red, siendo $throughput$ el volumen de información neta que fluye a través de una red. Este análisis nos permitirá definir que red es mas eficiente en términos de datos transportados.


\newpage
\section{Conclusiones}
En la introducción definimos dos fuentes de información $S$ y $S_{1}$, de las cuales, en la experimentación, calculamos su entropía y la información que provee cada símbolo. De esta manera pudimos distinguir determinados símbolos en cada fuente, según que Dataset utilizamos para el cálculo. A partir de ese cálculo, pudimos concluir que:
\begin{itemize}
  \item \textbf{Caso fuente $S$, protocolos}: Sin importar el Dataset, el protocolo \texttt{IP} se distingue claramente de los demás. Vale la pena mencionar el caso del protocolo \texttt{IPv6}, cuya probabilidad de ocurrencia es baja, lo cual da una pauta de su baja implementación.
  \item \textbf{Caso fuente $S_{1}$, nodos}: El Dataset juega un rol diferenciador a la hora de distinguir nodos, donde, según el Dataset, podemos observar uno o mas nodos distinguidos. Este fenómeno se explica por la topología de cada red, lo cual puede contener más de un gateway.
  Otra observación que se desprende sobre la topología de la redes es que su forma es de tipo estrella, lo cual condice con las metodologías de ruteo explicadas en clase.
\end{itemize}

Con respecto a la incidencia de los paquetes \texttt{ARP} en las diversas redes, pudimos observar que 
%existe una correlación entre la cantidad de nodos de la red y el porcentaje de paquetes \texttt{ARP}, siendo la misma proporcional. La razón de este fenómeno (especulamos) se debe a que, a mayor cantidad de nodos, mayor son los paquetes solicitando la dirección \texttt{MAC} del gateway y mayor es la probabilidad de que haya comunicaciones entre los diversos nodos de la red local. Dado que a mayor incidencia de paquetes \texttt{ARP} menor es el $throughput$, notamos un problema de escalabilidad de los protocolos vigentes, lo cual es producto de razones históricas, cuyo desenlace fue una proliferación de protocolos de capa 2, los cuales se buscaron homogeneizar con el protocolo \texttt{IP}, siendo necesario para tal fin, la introducción de los paquetes \texttt{ARP}. Entendemos que proponer/implementar otro protocolo de capa 2 y 3 es inviable y claramente escapa el alcance del presente trabajo.


% \newpage
% \bibliographystyle{plain}
% \section{Referencias}
% \begingroup
% \renewcommand{\section}[2]{}
% \bibliography{informe}
% \endgroup
%
% \newpage
% \appendix
% \input{apendice}

\end{document}
