\documentclass[a4paper]{article}
\usepackage[spanish]{babel}
\usepackage[utf8]{inputenc}
\usepackage{fancyhdr}
\usepackage{charter} % tipografia
%\usepackage{graphicx}
\usepackage[pdftex]{graphicx}
\usepackage{bm} % bold font in math mode
\usepackage{sidecap}
\usepackage{caption}
\usepackage{subcaption}
\usepackage{booktabs}
\usepackage{makeidx}
\usepackage{float}
\usepackage{amsmath, amsthm, amssymb}
\newtheorem{theorem}{Teorema}
\newtheorem{customthm}{Teorema}
\newtheorem{corollary}{Corolario}[theorem]
\newtheorem{proposition}[theorem]{Proposición}
\newtheorem{innercustomlemma}{Lemma}
\newenvironment{customlemma}[1]
  {\renewcommand\theinnercustomlemma{#1}\innercustomlemma}
  {\endinnercustomlemma}
\usepackage{amsfonts}
\usepackage{sectsty}
\usepackage{wrapfig}
\usepackage{listings}
\usepackage{hyperref} % links
\usepackage{algorithm} %http://www.ctan.org/pkg/algorithms
\usepackage{algorithmic}
\usepackage[usenames,dvipsnames]{xcolor}
\usepackage{pgfplots}
\usepackage{tabularx} % tablas copadas
% \usepackage{pgfplotstable}
% custom
\input{codesnippet}
\input{page.layout}
\input{extras}
% \setcounter{secnumdepth}{2}
\usepackage{underscore}
\usepackage{kbordermatrix}% Matrix column labels
\usetikzlibrary{arrows,shapes}
\usepackage{tkz-graph}
\usepackage{caratula}
\usepackage{url}
\lstset{
    language=C++,
    basicstyle=\ttfamily,
    keywordstyle=\color{blue}\ttfamily,
    stringstyle=\color{red}\ttfamily,
    commentstyle=\color{ForestGreen}\ttfamily,
    morecomment=[l][\color{magenta}]{\#},
    literate={á}{{\'a}}1 {ó}{{\'o}}1 {é}{{\'e}}1 {í}{{\'i}}1 {ú}{{\'u}}1 {Á}{{\'A}}1 {Í}{{\'I}}1 {É}{{\'E}}1 {Ú}{{\'U}}1 {Ó}{{\'O}}1 {\ \ }{{\ }}1,
	breaklines=true,
	tabsize=2
}

\DeclareUnicodeCharacter{2212}{-}

% *********************** %
\usepackage{tikz}
\usetikzlibrary{graphs}
\usetikzlibrary{calc}
\usetikzlibrary{arrows}
\usetikzlibrary{matrix}
% Otros
\usepackage{arrayjobx}
\usepackage{enumitem}
\usepackage{multicol}
\usepackage{natbib}
\usepackage{etoolbox}
\usepackage{listingsutf8}
\lstset{inputencoding=utf8/latin1}
\usepackage{fancyvrb}
\usepackage{pgfplotstable}
\usepackage{float}
\newcommand{\subscript}[2]{$#1 _ #2$}


% ******************************************************** %
\begin{document}
\thispagestyle{empty}
\materia{Teoría de las Comunicaciones}
\submateria{Primer Cuatrimestre de 2016}
\titulo{Trabajo Práctico I}
%\subtitulo{Grupo: }
\integrante{Iv\'an Arcuschin}{678/13}{iarcuschin@gmail.com}
\integrante{Federico De Rocco}{408/13}{fede.183@hotmail.com}
\integrante{Mart\'in Jedwabny}{885/13}{martiniedva@gmail.com}
\integrante{Jos\'e Massigoge}{954/12}{jmmassigoge@gmail.com}
\maketitle
% no footer on the first page
\thispagestyle{empty}
\newpage

\tableofcontents

\newpage
\section{Introducción}
La problemática de este trabajo práctico fue implementar una herramienta para escuchar a la red y analizar los datos que obteníamos. A la vez, con esos datos propusimos diferentes
fuentes de información para las cuales calculamos la incidencia de cada símbolo de la fuente(mayoritariamente distinguidos por protocolos y nodos) y la entropía entre otras 
cosas. 

Usamos el lenguaje de programación Python y su librería Scapy para llevar a cabo el sniffeo de la red. 

Aprovechamos estas herramientas para realizar varios experimentos en distintos lugares y condiciones de las redes. A partir de los datos adquiridos generamos gráficos que representan
el comportamiento de la red y discutimos sus características.

\newpage
\section{Experimentos}
%En esta sección desarrollaremos y mostraremos los resultados de los experimentos para:

\begin{itemize}
\item Los protocolos distinguidos.
\item La incidencia de paquetes ARP en la red.
\item Los nodos distinguidos.
\end{itemize}

\section{Experimento incidencia de ARP}

Para poder observar la incidencia de paqueres ARP en la red voy a utilizar una serie de muestras tomadas de distintas redes y graficarlas en un histograma, despues de mostrar
los resultados daremos una explicación del por que de los mismo.

\begin{figure}[ht!]
\centering
\includegraphics[width=90mm]{imagenes/IncidenciaARP.jpg}
\caption{Comparación de porcentaje de apariciones de ARP en distintas redes o con distintas condiciones.\label{overflow}}
\end{figure}

Como podemos ver la red de los laboratorios tiene más porcentaje de paquetes ARP que el de la biblioteca. Esto se debe a que en el laboratorio del pabellón 1 ingresan 
constantemente personas y, por lo tanto, la mayoría de ellos ingresan en la red a través de las computadoras del mismo o de usando el wifi de estos con sus celulares. Esto 
ocurre en menor medida en la biblioteca ya que no se usan computadoras que no sean propias de los que ingresan, por lo tanto, por la red hay mucho más trafico de paquetes 
ARP en los laboratorios. Además al estar en una red privada, es más común que se efectúen envíos de mensajes entre computadoras y los routers de las mismas.

\begin{comment}
Intento de experimentos1:

Para poder observar la incidencia de paqueres ARP en la red voy a utilizar el buscador de Google Chrome y la función de scapy arping. Usando la interfaz de wlan0 podre sniffear
los paquetes que habra con el buscador y usando arping podre enviar paquetes ARP.\\

Ejecuto  sudo ./capture.py wlan0 entropia-tipos 1000 y, al mismo tiempo, abro el buscador. 

Resultados:

Simbolo 2048 tiene probabilidad 0.996158770807(IP).

Simbolo 34525 tiene probabilidad 0.00128040973111(IPv6).

Simbolo 2054 tiene probabilidad 0.00256081946223(ARP).

La entropia de la fuente es 0.0492085855963.

Simbolo 2048 tiene probabilidad 0.996158770807(IP).

Simbolo 34525 tiene probabilidad 0.00256081946223(IPv6).

Simbolo 2054 tiene probabilidad 0.00128040973111(ARP).

La entropia de la fuente es 0.0398813035719.

Simbolo 2048 tiene probabilidad 0.99875(IP).

Simbolo 2054 tiene probabilidad 0.00125(ARP).

La entropia de la fuente es 0.0138570614629.

Simbolo 2048 tiene probabilidad 0.998740554156(IP).

Simbolo 2054 tiene probabilidad 0.00125944584383(ARP).

La entropia de la fuente es 0.013948087353.

Simbolo 2048 tiene probabilidad 0.985987261146(IP).

Simbolo 34525 tiene probabilidad 0.0101910828025(IPv6).

Simbolo 2054 tiene probabilidad 0.00382165605096(ARP).

La entropia de la fuente es 0.118197558737.

Ejecuto  sudo ./capture.py wlan0 entropia-tipos 100 y, al mismo tiempo, ejecuto arping("192.168.2.0/24").

Resultados:
Simbolo 2048 tiene probabilidad 0.0481481481481(IP).

Simbolo 2054 tiene probabilidad 0.951851851852(ARP).

La entropia de la fuente es 0.278477724908.

Simbolo 2048 tiene probabilidad 0.0769230769231(IP).

Simbolo 34525 tiene probabilidad 0.0244755244755(IPv6).

Simbolo 2054 tiene probabilidad 0.898601398601(ARP).

La entropia de la fuente es 0.554261234655.

Simbolo 2048 tiene probabilidad 0.0115384615385(IP).

Simbolo 2054 tiene probabilidad 0.988461538462(ARP).

La entropia de la fuente es 0.0908278259323.

Simbolo 2048 tiene probabilidad 0.011320754717(IP).

Simbolo 34525 tiene probabilidad 0.00754716981132(IPv6).

Simbolo 2054 tiene probabilidad 0.981132075472(ARP).

La entropia de la fuente es 0.153356025339.

Simbolo 2054 tiene probabilidad 1.0(ARP).

La entropia de la fuente es 0.0.

Ejecuto  sudo ./capture.py wlan0 entropia-tipos 100 y, al mismo tiempo, abro buscador y ejecuto arping("192.168.2.0/24").

Resultados:

Simbolo 2048 tiene probabilidad 0.737424547284(IP).

Simbolo 34525 tiene probabilidad 0.00201207243461(IPv6).

Simbolo 2054 tiene probabilidad 0.260563380282(ARP).

La entropia de la fuente es 0.847640262877.

Simbolo 2048 tiene probabilidad 0.735412474849(IP).

Simbolo 34525 tiene probabilidad 0.00603621730382(IPv6).

Simbolo 2054 tiene probabilidad 0.258551307847(ARP).

La entropia de la fuente es 0.875119992479.

Simbolo 2048 tiene probabilidad 0.761948529412(IP).

Simbolo 34525 tiene probabilidad 0.00183823529412(IPv6).

Simbolo 2054 tiene probabilidad 0.236213235294(ARP).

La entropia de la fuente es 0.807325191625.

Simbolo 2048 tiene probabilidad 0.757518796992(IP).

Simbolo 34525 tiene probabilidad 0.00093984962406(IPv6).

Simbolo 2054 tiene probabilidad 0.241541353383(ARP).

La entropia de la fuente es 0.808024777478.

Simbolo 2048 tiene probabilidad 0.754066985646(IP).

Simbolo 2054 tiene probabilidad 0.245933014354(ARP).

La entropia de la fuente es 0.804768238572.


Ejecuto  sudo ./capture.py wlan0 entropia-tipos 100 y, al mismo tiempo, arping("192.168.2.0/24") y arping("10.4.2.0/27").
\end{comment}
En esta sección desarrollaremos y mostraremos los resultados de los experimentos para:

\begin{itemize}
\item Los protocolos distinguidos.
\item La incidencia de paquetes ARP en la red.
\item Los nodos distinguidos.
\end{itemize}

Intento de experimentos1:

Para poder observar la incidencia de paqueres ARP en la red voy a utilizar el buscador de Google Chrome y la función de scapy arping. Usando la interfaz de wlan0 podre sniffear
los paquetes que habra con el buscador y usando arping podre enviar paquetes ARP.\\

Ejecuto  sudo ./capture.py wlan0 entropia-tipos 1000 y, al mismo tiempo, abro el buscador. 

Resultados:

Simbolo 2048 tiene probabilidad 0.996158770807(IP).

Simbolo 34525 tiene probabilidad 0.00128040973111(IPv6).

Simbolo 2054 tiene probabilidad 0.00256081946223(ARP).

La entropia de la fuente es 0.0492085855963.

Simbolo 2048 tiene probabilidad 0.996158770807(IP).

Simbolo 34525 tiene probabilidad 0.00256081946223(IPv6).

Simbolo 2054 tiene probabilidad 0.00128040973111(ARP).

La entropia de la fuente es 0.0398813035719.

Simbolo 2048 tiene probabilidad 0.99875(IP).

Simbolo 2054 tiene probabilidad 0.00125(ARP).

La entropia de la fuente es 0.0138570614629.

Simbolo 2048 tiene probabilidad 0.998740554156(IP).

Simbolo 2054 tiene probabilidad 0.00125944584383(ARP).

La entropia de la fuente es 0.013948087353.

Simbolo 2048 tiene probabilidad 0.985987261146(IP).

Simbolo 34525 tiene probabilidad 0.0101910828025(IPv6).

Simbolo 2054 tiene probabilidad 0.00382165605096(ARP).

La entropia de la fuente es 0.118197558737.

Ejecuto  sudo ./capture.py wlan0 entropia-tipos 100 y, al mismo tiempo, ejecuto arping("192.168.2.0/24").

Resultados:
Simbolo 2048 tiene probabilidad 0.0481481481481(IP).

Simbolo 2054 tiene probabilidad 0.951851851852(ARP).

La entropia de la fuente es 0.278477724908.

Simbolo 2048 tiene probabilidad 0.0769230769231(IP).

Simbolo 34525 tiene probabilidad 0.0244755244755(IPv6).

Simbolo 2054 tiene probabilidad 0.898601398601(ARP).

La entropia de la fuente es 0.554261234655.

Simbolo 2048 tiene probabilidad 0.0115384615385(IP).

Simbolo 2054 tiene probabilidad 0.988461538462(ARP).

La entropia de la fuente es 0.0908278259323.

Simbolo 2048 tiene probabilidad 0.011320754717(IP).

Simbolo 34525 tiene probabilidad 0.00754716981132(IPv6).

Simbolo 2054 tiene probabilidad 0.981132075472(ARP).

La entropia de la fuente es 0.153356025339.

Simbolo 2054 tiene probabilidad 1.0(ARP).

La entropia de la fuente es 0.0.

Ejecuto  sudo ./capture.py wlan0 entropia-tipos 100 y, al mismo tiempo, abro buscador y ejecuto arping("192.168.2.0/24").

Resultados:

Simbolo 2048 tiene probabilidad 0.737424547284(IP).

Simbolo 34525 tiene probabilidad 0.00201207243461(IPv6).

Simbolo 2054 tiene probabilidad 0.260563380282(ARP).

La entropia de la fuente es 0.847640262877.

Simbolo 2048 tiene probabilidad 0.735412474849(IP).

Simbolo 34525 tiene probabilidad 0.00603621730382(IPv6).

Simbolo 2054 tiene probabilidad 0.258551307847(ARP).

La entropia de la fuente es 0.875119992479.

Simbolo 2048 tiene probabilidad 0.761948529412(IP).

Simbolo 34525 tiene probabilidad 0.00183823529412(IPv6).

Simbolo 2054 tiene probabilidad 0.236213235294(ARP).

La entropia de la fuente es 0.807325191625.

Simbolo 2048 tiene probabilidad 0.757518796992(IP).

Simbolo 34525 tiene probabilidad 0.00093984962406(IPv6).

Simbolo 2054 tiene probabilidad 0.241541353383(ARP).

La entropia de la fuente es 0.808024777478.

Simbolo 2048 tiene probabilidad 0.754066985646(IP).

Simbolo 2054 tiene probabilidad 0.245933014354(ARP).

La entropia de la fuente es 0.804768238572.


Ejecuto  sudo ./capture.py wlan0 entropia-tipos 100 y, al mismo tiempo, arping("192.168.2.0/24") y arping("10.4.2.0/27").

\newpage
\section{Conclusiones}
\input{conclusiones}

% \newpage
% \bibliographystyle{plain}
% \section{Referencias}
% \begingroup
% \renewcommand{\section}[2]{}
% \bibliography{informe}
% \endgroup
%
% \newpage
% \appendix
% \input{apendice}

\end{document}
